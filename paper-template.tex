\def\grammarly{off}

\def\paperversiondraft{draft}
\def\paperversionblind{blind}
\def\paperversioncamera{camera}
\ifx\paperversion\paperversiondraft
\else
  \ifx\paperversion\paperversionblind
  \else
    \ifx\paperversion\paperversioncamera
    \else
       \errmessage{paperversion must be set to 'draft', 'blind', or 'camera'}
    \fi
  \fi
\fi

\def\grammarlyon{on}
\def\grammarlyoff{off}

\ifx\grammarly\grammarlyon
\def\review{}
\else
\def\review{review,}
\fi

\ifx\paperversion\paperversiondraft
  \documentclass[sigplan,\review anonymous]{acmart}
\fi

\ifx\paperversion\paperversionblind
  \documentclass[sigplan,\review anonymous]{acmart}
\fi

\ifx\paperversion\paperversioncamera
  \documentclass[sigplan]{acmart}\settopmatter{}
\fi

\usepackage{xargs}
\usepackage{todonotes}
\usepackage{xparse}
\usepackage{xifthen, xstring}
\usepackage{ulem}
\usepackage{xspace}
\makeatletter
\font\uwavefont=lasyb10 scaled 652

\newcommand\colorwave[1][blue]{\bgroup\markoverwith{\lower3\p@\hbox{\uwavefont\textcolor{#1}{\char58}}}\ULon}
\makeatother

\ifx\paperversion\paperversiondraft
\newcommand\createtodoauthor[2]{%
\def\tmpdefault{emptystring}
\expandafter\newcommand\csname #1\endcsname[2][\tmpdefault]{\def\tmp{##1}\ifthenelse{\equal{\tmp}{\tmpdefault}}
   {\todo[linecolor=#2!20,backgroundcolor=#2!25,bordercolor=#2]{A \textbf{#1:} ##2}}
   {\ifthenelse{\equal{##2}{}}{\colorwave[#2]{##1}\xspace}{ \todo[linecolor=#2!10,backgroundcolor=#2!25,bordercolor=#2]{\textbf{#1:} ##2}\colorwave[#2]{##1}}}}}
\else
\newcommand\createtodoauthor[2]{%
\expandafter\newcommand\csname #1\endcsname[2][\@nil]{}}
\fi

% Broaden margins to make room for todo notes
\ifx\paperversion\paperversiondraft
  \paperwidth=\dimexpr \paperwidth + 6cm\relax
  \oddsidemargin=\dimexpr\oddsidemargin + 3cm\relax
  \evensidemargin=\dimexpr\evensidemargin + 3cm\relax
  \marginparwidth=\dimexpr \marginparwidth + 3cm\relax
  \setlength{\marginparwidth}{4cm}
\fi

\createtodoauthor{authorOne}{red}
\createtodoauthor{authorTwo}{blue}
\createtodoauthor{authorThree}{green}
\createtodoauthor{authorFour}{orange}
\createtodoauthor{authorFive}{purple}

%% Note: Authors migrating a paper from traditional SIGPLAN
%% proceedings format to PACMPL format should change 'sigplan' to
%% 'acmsmall'.


%% Some recommended packages.
\usepackage{booktabs}   %% For formal tables:
                        %% http://ctan.org/pkg/booktabs
\usepackage{subcaption} %% For complex figures with subfigures/subcaptions
                        %% http://ctan.org/pkg/subcaption


\makeatletter\if@ACM@journal\makeatother
%% Journal information (used by PACMPL format)
%% Supplied to authors by publisher for camera-ready submission
\acmJournal{PACMPL}
\acmVolume{1}
\acmNumber{1}
\acmArticle{1}
\acmYear{2017}
\acmMonth{1}
\acmDOI{10.1145/nnnnnnn.nnnnnnn}
\startPage{1}
\else\makeatother
%% Conference information (used by SIGPLAN proceedings format)
%% Supplied to authors by publisher for camera-ready submission
\acmConference[PL'17]{ACM SIGPLAN Conference on Programming Languages}{January 01--03, 2017}{New York, NY, USA}
\acmYear{2017}
\acmISBN{978-x-xxxx-xxxx-x/YY/MM}
\acmDOI{10.1145/nnnnnnn.nnnnnnn}
\startPage{1}
\fi


%% Copyright information
%% Supplied to authors (based on authors' rights management selection;
%% see authors.acm.org) by publisher for camera-ready submission
\setcopyright{none}             %% For review submission
%\setcopyright{acmcopyright}
%\setcopyright{acmlicensed}
%\setcopyright{rightsretained}
%\copyrightyear{2017}           %% If different from \acmYear


%% Bibliography style
\bibliographystyle{ACM-Reference-Format}
%% Citation style
%% Note: author/year citations are required for papers published as an
%% issue of PACMPL.
%\citestyle{acmauthoryear}  %% For author/year citations
%\citestyle{acmnumeric}     %% For numeric citations
%\setcitestyle{nosort}      %% With 'acmnumeric', to disable automatic
                            %% sorting of references within a single citation;
                            %% e.g., \cite{Smith99,Carpenter05,Baker12}
                            %% rendered as [14,5,2] rather than [2,5,14].
%\setcitesyle{nocompress}   %% With 'acmnumeric', to disable automatic
                            %% compression of sequential references within a
                            %% single citation;
                            %% e.g., \cite{Baker12,Baker14,Baker16}
                            %% rendered as [2,3,4] rather than [2-4].



\begin{document}

%% Title information
\title[Short Title]{Full Title}         %% [Short Title] is optional;
                                        %% when present, will be used in
                                        %% header instead of Full Title.
\titlenote{with title note}             %% \titlenote is optional;
                                        %% can be repeated if necessary;
                                        %% contents suppressed with 'anonymous'
\subtitle{Subtitle}                     %% \subtitle is optional
\subtitlenote{with subtitle note}       %% \subtitlenote is optional;
                                        %% can be repeated if necessary;
                                        %% contents suppressed with 'anonymous'


%% Author information
%% Contents and number of authors suppressed with 'anonymous'.
%% Each author should be introduced by \author, followed by
%% \authornote (optional), \orcid (optional), \affiliation, and
%% \email.
%% An author may have multiple affiliations and/or emails; repeat the
%% appropriate command.
%% Many elements are not rendered, but should be provided for metadata
%% extraction tools.

%% Author with single affiliation.
\author{First1 Last1}
\authornote{with author1 note}          %% \authornote is optional;
                                        %% can be repeated if necessary
\orcid{nnnn-nnnn-nnnn-nnnn}             %% \orcid is optional
\affiliation{
  \position{Position1}
  \department{Department1}              %% \department is recommended
  \institution{Institution1}            %% \institution is required
  \streetaddress{Street1 Address1}
  \city{City1}
  \state{State1}
  \postcode{Post-Code1}
  \country{Country1}
}
\email{first1.last1@inst1.edu}          %% \email is recommended

%% Author with two affiliations and emails.
\author{First2 Last2}
\authornote{with author2 note}          %% \authornote is optional;
                                        %% can be repeated if necessary
\orcid{nnnn-nnnn-nnnn-nnnn}             %% \orcid is optional
\affiliation{
  \position{Position2a}
  \department{Department2a}             %% \department is recommended
  \institution{Institution2a}           %% \institution is required
  \streetaddress{Street2a Address2a}
  \city{City2a}
  \state{State2a}
  \postcode{Post-Code2a}
  \country{Country2a}
}
\email{first2.last2@inst2a.com}         %% \email is recommended
\affiliation{
  \position{Position2b}
  \department{Department2b}             %% \department is recommended
  \institution{Institution2b}           %% \institution is required
  \streetaddress{Street3b Address2b}
  \city{City2b}
  \state{State2b}
  \postcode{Post-Code2b}
  \country{Country2b}
}
\email{first2.last2@inst2b.org}         %% \email is recommended


%% Paper note
%% The \thanks command may be used to create a "paper note" ---
%% similar to a title note or an author note, but not explicitly
%% associated with a particular element.  It will appear immediately
%% above the permission/copyright statement.
\thanks{with paper note}                %% \thanks is optional
                                        %% can be repeated if necesary
                                        %% contents suppressed with 'anonymous'


%% Abstract
%% Note: \begin{abstract}...\end{abstract} environment must come
%% before \maketitle command
\begin{abstract}
Text of abstract \ldots.
\end{abstract}


% Only add ACM notes and keywords in camera ready version
% Drop citations and footnotes in draft and blind mode.
\ifx\paperversion\paperversioncamera
%% 2012 ACM Computing Classification System (CSS) concepts
%% Generate at 'http://dl.acm.org/ccs/ccs.cfm'.
\begin{CCSXML}
<ccs2012>
<concept>
<concept_id>10011007.10011006.10011008</concept_id>
<concept_desc>Software and its engineering~General programming languages</concept_desc>
<concept_significance>500</concept_significance>
</concept>
<concept>
<concept_id>10003456.10003457.10003521.10003525</concept_id>
<concept_desc>Social and professional topics~History of programming languages</concept_desc>
<concept_significance>300</concept_significance>
</concept>
</ccs2012>
\end{CCSXML}

\ccsdesc[500]{Software and its engineering~General programming languages}
\ccsdesc[300]{Social and professional topics~History of programming languages}
%% End of generated code


%% Keywords
%% comma separated list
\keywords{keyword1, keyword2, keyword3}  %% \keywords is optional

\else
\settopmatter{printacmref=false} % Removes citation information below abstract
\renewcommand\footnotetextcopyrightpermission[1]{} % removes footnote with conference information in first column
\fi

%% \maketitle
%% Note: \maketitle command must come after title commands, author
%% commands, abstract environment, Computing Classification System
%% environment and commands, and keywords command.
\maketitle
\ifx\grammarly\grammarlyon 
\onecolumn 
\else 
\fi

\section{Introduction}

Of this super cool article \ldots
\authorOne{Comment}

Hallo this is a long \authorOne[Text]{Comment} that is interesting.

And here we only underline \authorOne[Text]{} that is interesting.

Now we have a sequence of comments,
\authorOne{Comment 1}
\authorTwo{Comment 2}
\authorThree{Comment 3}
\authorFour{Comment 4}
\authorFive{Comment 5}
which should not introduce overlong white space sequences.

%% Acknowledgments
\begin{acks}                            %% acks environment is optional
                                        %% contents suppressed with 'anonymous'
  %% Commands \grantsponsor{<sponsorID>}{<name>}{<url>} and
  %% \grantnum[<url>]{<sponsorID>}{<number>} should be used to
  %% acknowledge financial support and will be used by metadata
  %% extraction tools.
  This material is based upon work supported by the
  \grantsponsor{GS100000001}{National Science
    Foundation}{http://dx.doi.org/10.13039/100000001} under Grant
  No.~\grantnum{GS100000001}{nnnnnnn} and Grant
  No.~\grantnum{GS100000001}{mmmmmmm}.  Any opinions, findings, and
  conclusions or recommendations expressed in this material are those
  of the author and do not necessarily reflect the views of the
  National Science Foundation.
\end{acks}


%% Bibliography
%\bibliography{bibfile}


%% Appendix
\appendix
\section{Appendix}

Text of appendix \ldots

\end{document}
